% Created 2022-11-04 Παρ 14:10
% Intended LaTeX compiler: pdflatex
\documentclass[11pt]{article}
\usepackage[utf8]{inputenc}
\usepackage[T1]{fontenc}
\usepackage{graphicx}
\usepackage{longtable}
\usepackage{wrapfig}
\usepackage{rotating}
\usepackage[normalem]{ulem}
\usepackage{amsmath}
\usepackage{amssymb}
\usepackage{capt-of}
\usepackage{hyperref}
\author{Τοροσιάν Νικόλας ΤΜ6220}
\date{\today}
\title{Μέτρηση Ωμικής Αντίστασης, ρεύματος και τάσης}
\hypersetup{
 pdfauthor={Τοροσιάν Νικόλας ΤΜ6220},
 pdftitle={Μέτρηση Ωμικής Αντίστασης, ρεύματος και τάσης},
 pdfkeywords={},
 pdfsubject={},
 pdfcreator={Emacs 28.2 (Org mode 9.6)}, 
 pdflang={English}}
\begin{document}

\maketitle
\tableofcontents

\section{Άσκηση 1}
\label{sec:orga324e3f}
Δίνεται κύκλωμα όπου R1=1 kΩ, R2=1.5 kΩ, R3=680 Ω, R4=220 Ω

Μετρήσεις Εργαστηριακής εγκατάστασης
\begin{center}
\begin{tabular}{rrrrrrrr}
Vr1 & Vr2 & Vr3 & Vr4 & I1 & I2 & I3 & I4\\
7.63 & 4.36 & 3.3 & 1.69 & 7.83 & 2.96 & 4.91 & 4.91\\
\end{tabular}
\end{center}

Μετρήσεις Tinkercad
\begin{center}
\begin{tabular}{rrrrrrrr}
Vr1 & Vr2 & Vr3 & Vr4 & I1 & I2 & I3 & I4\\
7.68 & 4.32 & 3.26 & 1.06 & 7.68 & 2.88 & 4.80 & 4.80\\
\end{tabular}
\end{center}

Μετρήσεις Θεωριτικής επίλυσης κυκλώματος

Kirchhoff \(1_s_t\) law

\[\sum_{n=1}^{n} I_n = 0\]

Kirchhoff \(2_s_t\) law

\[\sum_{n=1}^{n} V_n = 0\]

\begin{center}
\begin{tabular}{llllllll}
Vr1 & Vr2 & Vr3 & Vr4 & I1 & I2 & I3 & I4\\
 &  &  &  &  &  &  & \\
\end{tabular}
\end{center}


\section{Άσκηση 2}
\label{sec:org6c789ed}
Δίνεται κύκλωμα όπου R1=1 kΩ, R2=1.5 kΩ, R3=680 Ω, R4=220 Ω, R5=1 kΩ, R6=1 kΩ

Μετρήσεις Εργαστηριακής εγκατάστασης
\begin{center}
\begin{tabular}{llllllllllll}
Vr1 & Vr2 & Vr3 & Vr4 & Vr5 & Vr6 & I1 & I2 & I3 & I6 & I5 & I6\\
 &  &  &  &  &  &  &  &  &  &  & \\
\end{tabular}
\end{center}

Μετρήσεις Tinkercad
\begin{center}
\begin{tabular}{llllllllllll}
Vr1 & Vr2 & Vr3 & Vr4 & Vr5 & Vr6 & I1 & I2 & I3 & I6 & I5 & I6\\
 &  &  &  &  &  &  &  &  &  &  & \\
\end{tabular}
\end{center}

Μετρήσεις Θεωριτικής επίλυσης κυκλώματος
\begin{center}
\begin{tabular}{llllllllllll}
Vr1 & Vr2 & Vr3 & Vr4 & Vr5 & Vr6 & I1 & I2 & I3 & I6 & I5 & I6\\
 &  &  &  &  &  &  &  &  &  &  & \\
\end{tabular}
\end{center}
\end{document}
